\documentclass[aps,showpacs,twocolumn,floats,prd,superscriptaddress,nofootinbib]{revtex4} 
\usepackage{graphicx,amsmath,amssymb,amstext}
\usepackage{amssymb,amsbsy,amsfonts,amsthm,color}

\usepackage{epsfig}
%\usepackage{showkeys}
\usepackage{graphicx}
\usepackage{subfigure}

\graphicspath{{Figures/}}

\begin{document}

\title{Predicting the gravitational polarization rotation of the binary pulsar}

\author{Ue-Li Pen}
\email{pen@cita.utoronto.ca}
\affiliation{Canadian Institute of Theoretical Astrophysics, 60 St George St, Toronto, ON M5S 3H8, Canada.}
\affiliation{Canadian Institute for Advanced Research, CIFAR program in Gravitation and Cosmology.}
\affiliation{Dunlap Institute for Astronomy \& Astrophysics, University of Toronto, AB 120-50 St. George Street, Toronto, ON M5S 3H4, Canada.}
\affiliation{Perimeter Institute of Theoretical Physics, 31 Caroline Street North, Waterloo, ON N2L 2Y5, Canada.}

\author{Xin Wang}
\email{xwang@cita.utoronto.ca}
\affiliation{Canadian Institute of Theoretical Astrophysics, 60 St George St, Toronto, ON M5S 3H8, Canada.}

\author{I-Sheng Yang}
\email{isheng.yang@gmail.com}
\affiliation{Canadian Institute of Theoretical Astrophysics, 60 St George St, Toronto, ON M5S 3H8, Canada.}
\affiliation{Perimeter Institute of Theoretical Physics, 31 Caroline Street North, Waterloo, ON N2L 2Y5, Canada.}

\begin{abstract}
Some abstract stuffs.
\end{abstract}

\maketitle

\section{Introduction}

Einstein's theory of General Relativity has been the dominant theory of gravity for a century. Many of its signature outcomes, such as light-bending and gravitational waves, have been confirmed by precision tests from both astronomy and cosmology. One last pending test is the gravitational rotation of polarizations. Indeed, since gravity directly affects the spacetime geometry, not only the paths of light can be bent, but also the polarization, intuitively an ``internal'' degree of freedom of the light ray, cannot escape the influence of gravity.

Such gravitational rotation of polarization has not been measured so far.  One obvious reason is that it is usually very small. The rotation angle of a linearly polarized light ray, caused by a gravitational lens, is suppressed by two small numbers.
\begin{equation}
\Delta\phi \propto \frac{2GM}{r} \cdot v~.
\label{eq-v}
\end{equation}
Here $M$ is the mass of the lens, $r$ is the impact parameter---the shortest distance when the light ray pass near the lens, and $v$ is the velocity of the lens.  Unless the light goes through somewhere comparable to the Schwarzschild radius, the first factor is small. Unless the velocity is almost relativistic, the second factor is small. 

Luckily, the double pulsar PSR J0737-3039 is an ideal candidate to measure such effect. The pulsar signal is highly polarized, allowing precise measurements of its rotation; its almost edge-on orbit allows the impact parameter to be very small at the superior conjunction; and its compact orbit means a large velocity. The main point of this paper is to show that one can expect to have $\Delta\phi\sim 10^{-7}$ from double pulsar, which has a good chance to be observable given a realistic observation campaign.

Gravitational rotation of polarization from double pulsar was previously studied in \cite{RugTar06}. They however predicted a much smaller number. We will first clarify such difference by providing a simple derivation of Eq.~(\ref{eq-v}). Although the derivation in \cite{KopMas01} clearly supports its validity, there has been confusions in the literature. In particular, there was some doubt about its gauge-invariance, since a boost and rotation can leave the light-ray invariant while reducing the velocity to zero, which should have zero effect. We will resolve such confusion by showing a gauge-invariant $SO(3,1)$ gravitational rotation of the tangent space along a light ray, and the rotation of polarization is an $SO(2)$ projection of that. The null vector of the light ray and the timelike vector of the observer together determines which $SO(2)$ to project to, {\it thus the actual rotation of polarization depends on the relative velocity between the lens and the observer,} and there is nothing surprising about that.

In Sec.\ref{sec-born}, we will derive Eq.~(\ref{eq-v}) in the Born approximation and the geometric limit. In Sec.\ref{sec-prediction}, we will describe the observable effects on double pulsar and discuss the appropriate observation campaign to detect it.

\section{Rotation of Polarization}
\label{sec-born}

\section{Double Pulsar}
\label{sec-prediction}

\bibliography{all_active}


\end{document}
