\documentclass[aps,showpacs,onecolumn,floats,prd,superscriptaddress,nofootinbib]{revtex4} 
\usepackage{graphicx,amsmath,amssymb,amstext}
\usepackage{amssymb,amsbsy,amsfonts,amsthm,color}

\usepackage{epsfig}
%\usepackage{showkeys}
\usepackage{graphicx}
\usepackage{subfigure}

\graphicspath{{Figures/}}

\begin{document}

\title{Predicting the gravitational polarization rotation of the binary pulsar}

\author{Ue-Li Pen}
\email{pen@cita.utoronto.ca}
\affiliation{Canadian Institute of Theoretical Astrophysics, 60 St George St, Toronto, ON M5S 3H8, Canada.}
\affiliation{Canadian Institute for Advanced Research, CIFAR program in Gravitation and Cosmology.}
\affiliation{Dunlap Institute for Astronomy \& Astrophysics, University of Toronto, AB 120-50 St. George Street, Toronto, ON M5S 3H4, Canada.}
\affiliation{Perimeter Institute of Theoretical Physics, 31 Caroline Street North, Waterloo, ON N2L 2Y5, Canada.}

\author{Xin Wang}
\email{xwang@cita.utoronto.ca}
\affiliation{Canadian Institute of Theoretical Astrophysics, 60 St George St, Toronto, ON M5S 3H8, Canada.}

\author{I-Sheng Yang}
\email{isheng.yang@gmail.com}
\affiliation{Canadian Institute of Theoretical Astrophysics, 60 St George St, Toronto, ON M5S 3H8, Canada.}
\affiliation{Perimeter Institute of Theoretical Physics, 31 Caroline Street North, Waterloo, ON N2L 2Y5, Canada.}

\begin{abstract}
Some abstract stuffs.
\end{abstract}

\maketitle

\section{Introduction}

Einstein's theory of General Relativity has been the dominant theory of gravity for a century. 
Many of its signature outcomes, such as light-bending, orbital precession, and gravitational waves, have been confirmed by precision tests from both astronomy and cosmology. 
{\bf XXX do we trouble ourselves to find those citations?}
One last pending test is the gravitational rotation of polarizations. 
Indeed, since gravity directly affects the spacetime geometry, not only the paths of light can be bent.  
Polarizations, often thought as ``internal'' degrees of freedom of the light ray, cannot escape the influence of gravity either.

Such gravitational rotation of polarization has not been measured so far.
\footnote{The E-mode and B-mode in CMB are defined as the relative angle between polarization and gradient. 
The observed rotation is a consequence of a rotated gradient but a fixed polarization, thus it does not count.} 
One obvious reason is that it is usually very small. 
The rotation angle of a linearly polarized light ray, caused by a gravitational lens, is suppressed by two small numbers.
\begin{equation}
\Delta\phi \approx \frac{4GM}{r} \cdot v~.
\label{eq-v}
\end{equation}
Here $M$ is the mass of the lens, $r$ is the impact parameter---the shortest distance when the light ray pass near the lens, and $v$ is the velocity of the lens.  
Unless the light goes through somewhere comparable to the Schwarzschild radius, the first factor is small. 
Unless the velocity is almost relativistic, the second factor is small. 

Luckily, an almost edge-on, compact pulsar binary system, such as the double pulsar PSR J0737-3039, can be a very strong candidate to measure such effect. 
Pulsar signals are often highly polarized, allowing precise measurements of its rotation; 
an almost edge-on orbit allows the impact parameter to be very small at the superior conjunction; 
its compact orbit means a large velocity. 
One main point of this paper is to show that one can expect to have $\Delta\phi\sim 10^{-7}$ from double pulsar, which has a good chance to be observable given a realistic observation campaign.

Gravitational rotation of polarization from double pulsar was previously studied in \cite{RugTar06}. 
They however derived a much smaller number which is incorrect. 
Such mistake is not difficult to understand. 
Although Eq.~(\ref{eq-v}) has been derived by some authors, such as in \cite{KopMas01}, a significant collection of literature is still filled with confusions. 
For example, \cite{BroDem11} summarized how three different values of $\Delta\phi$ can be derived from the same physical sitution, but it provided no explicit judgement, nor convincing reason, of which one is actually correct. 
There are also disagreements on ``whether there is a nonzero rotation in Schwarzschild metric'', which as we will show, is not even a legitimate question to ask.

In fact, even Eq.~(\ref{eq-v}) may appear to be confusing on its own.
It has an explicit dependence on $v$ which makes it manifestly not gauge-invariant at the leading order. 
In this paper, we will resolve the confusions and defend Eq.~(\ref{eq-v}). 
We will show that there is a unique, operationally meaningful definition of polarization rotation, which corresponds to an explicit measurement result, and is indeed given by Eq.~(\ref{eq-v}).
It turns out that there is a unique $SO(3,1)$ gravitational rotation of the tangent space along any light ray that starts and ends in asymptotic Minkowski space. 
However the rotation of polarization is an $SO(2)$ projection of that. 
The null vector of the light ray and the timelike vector of the observer together determines which $SO(2)$ to project to, {\it thus the actual rotation of polarization depends on the relative velocity between the lens and the observer,} which is exactly the $v$ dependence in Eq.~(\ref{eq-v}).
Therefore, one cannot simply ask whether there are rotations in Schwarzschild metric without specifying who the observer is.
For an observer at rest, there is indeed zero rotation.
For a moving observer, there will be a nonzero rotation.

In Sec.\ref{sec-born}, we will provide the operational definition of polarization from the basic principles in general relativity.
In Sec.\ref{sec-Sch}, we will derive Eq.~(\ref{eq-v}) in the Schwarzschild metric.
In Sec.\ref{sec-prediction}, we will describe the observable effects on double pulsar and discuss the appropriate observation campaign to detect it.

\section{Definition from Scratch}
\label{sec-born}

\subsection{Operational Definition}

Intuitively, one can imagine two linear polarizations as two vectors attached to a light ray. 
Let {\bf k} be the null vector of the light ray and {\bf e} be a polarization vector, a parallel transport of {\bf e} should be valid in the geometric optic limit.
\begin{equation}
k^a \nabla_a e^c = k^a \partial_a e^c + k^a e^b \Gamma_{ab}^c =0~.
\end{equation}
Thus when the rotation is small, using the Born approximation, integrating along a light ray from point A to point B leads to
\begin{equation}
\Delta e^c = \int_A^B \hat{k}^a e_0^b \Gamma_{ab}^c~dl~,
\label{eq-int}
\end{equation}
where ${\bf e_0}$ is the original vector and $\Delta${\bf e} is the change. 
Naturally, $\Delta \phi \equiv |\Delta {\bf e}|/|{\bf e}|$ is the straightforward definition of how much a polarization vector has been rotated.

This however, cannot be the full story. 
Eq.~(\ref{eq-int}) literally compares two vectors on the tangent spaces of two different points, which is meaningless. 
The two vectors must be in the same tangent space to provide a physically meaningful rotation. 
It turns out that Eq.~(\ref{eq-int}) does give the correct value, but some extra care is required to give it a clean definition.

First of all, parallel transport is not limited to null rays. 
We can have an integral similar to Eq.~(\ref{eq-int}) along any path. 
In particular, one can perform a loop integral, and the answer will be a meaningful comparison between two vectors on the same point. 
Secondly, any such loop integral gives zero in Minkowski space, thus one can define that any segment in Minkowski space contributes exactly zero to the rotation. 
Now if we have a light ray starts from point A and reaches point B, both in asymptotic Minkowski space, one can validate Eq.~(\ref{eq-int}) by adding a term to it.
\begin{equation}
\Delta e^c = \int_A^B \hat{k}^a e_0^b \Gamma_{ab}^c~dl +
 \int_B^A \hat{p}^a e_0^b \Gamma_{ab}^c dl~.
\label{eq-loop}
\end{equation}
The second term is a line integral from B back to A which stays in the asymptotic Minkowski part of the spacetime. 
It contributes exactly zero value, therefore it allows a well-defined loop integral to be assigned as the physical rotation of a line integral from A to B.

An actual observation works very similarly to Eq.~(\ref{eq-loop}). 
What we have is a source (pulsar) which constantly emits a fixed (albeit unknown) polarization. 
We measure the polarization during a usual time, which is a light ray from $A_1$ to $B_1$. 
And then we compare it with the polarization measured when its binary companion passes very close to the light of sight, which is another light ray from $A_2$ to $B_2$. 
We take the difference between these two measurements, which is exactly a loop integral.
\begin{equation}
\Delta e^c = \int_{A_2}^{B_2} \hat{k}^a e_0^b \Gamma_{ab}^c~dl +
\int_{B_2}^{B_1} \hat{p}^a e_0^b \Gamma_{ab}^c dl +
\int_{B_1}^{A_1} \hat{k}^a e_0^b \Gamma_{ab}^c~dl +
\int_{A_1}^{A_2} \hat{p}^a e_0^b \Gamma_{ab}^c dl~.
\label{eq-pulsar}
\end{equation}
The integral $B_1B_2$ and $A_1A_2$ are along timelike trajectories of the pulsar and the earth, which are effectively in the asymptotic region and nothing happens. 
The integral $A_1B_1$ is along a light ray without the influence of the companion. 
Thus the above loop integral is indeed calculating the rotation of polarization caused by the passage of the companion.

{\bf XXX need a figure here}

\subsection{Observer Dependence}

Eq.~(\ref{eq-int}) is the leading order effect of a small rotation matrix.
\begin{equation}
e^c = e_0^c + \Delta e^c = \Lambda^c_{\ b} e_0^b = 
\left( g^c_{\ b} + \Delta^c_{\ b} \right) e_0^b~,
\end{equation}
where
\begin{equation}
\Delta_{cb} = \int_A^B \hat{k}^a \Gamma^d_{ab} g_{cd}dl~.
\end{equation}
By definition of a rotation matrix, $\Delta_{cd}$ has to be anti-symmetric, which can be verified explicitly.
\begin{eqnarray}
\Delta_{ac} = \int k^b\Gamma_{ab}^d g_{cd}d\lambda &=& 
\frac{1}{2} \int k^b \left(\partial_a g_{bc} + \partial_bg_{ac} - \partial_c g_{ab}\right)d\lambda
\label{eq-Delta}
\\ \nonumber
&=& \frac{1}{2} \int k^b \left(\partial_a g_{bc} - \partial_c g_{ab}\right)d\lambda~.
\end{eqnarray}
Note that we have to drop the boundary term for this anti-symmetry, which is allowed because this is effectively a loop integral as we explained in the previous section.

This tells us that there is actually a full $SO(3,1)$ rotation, $\Lambda^a_{\ b}\approx\left(g^a_{\ b} + \Delta^a_{\ b}\right)$, that is associated with a light ray. 
This does not uniquely determine the polarization rotation, which is an $SO(2)$. 
It also contains extra information such as the deflection of the light ray itself. 
One needs to specify two polarization vectors to determine which $SO(2)$ to project to. 
For any observer, the polarization vectors are orthogonal to both the incoming light ray and its own worldline. 
Thus a projection to the co-dimension-two surface orthogonal to the light ray and the observer 4-velocity is the desired $SO(2)$ rotation of polarization. 
{\bf Therefore, it is natural and necessary that polarization rotation depends on the observer velocity}, which explains the $v$ dependence in Eq.~(\ref{eq-v}).

One last possible confusion is why such dependence is on the velocity of the observer instead of the source, since they seem to play equivalent roles in the integral of Eq.~(\ref{eq-int}). 
We remind the reader again that the apparent line integral in Eq.~(\ref{eq-int}) is a convenient illusion. 
The physically meaningful is always along a loop, where one sends out a polarization vector and waits for it to come back to see the difference. 
Thus there is one unique point at which the rotation is defined. 
In practice, we will have no idea about the actual polarization when the signal is emitted at the pulsar. 
All we know are the polarizations we received on earth, so that is the unique 4-velocity we care about.



\section{Explicit calculation}
\label{sec-Sch}

\subsection{Point Mass}

We will treat the gravitational lens as a point mass and model it with a Schwarzschild metric in the isotropic form, expanded to the leading order of $(M/r)$. 
The newton constant $G$ is conveniently set to 1.
\begin{eqnarray}
g_{ab}dx^adx^b &=& -\left(1-\frac{2M}{r}\right)dt^2 + \left(1+\frac{2M}{r}\right)\left(dx^2+dy^2+dz^2\right)~, \\
r^2 &=& x^2 + y^2 + z^2~.
\label{eq-SchIso}
\end{eqnarray}
Instead of studying an arbitrary light ray in the above coordinate, we will shift and boost the above metric such that the lens has arbitrary position and velocity, and the relevant light ray is aways $x=t$.
This means six parameters, $(x_0,y_0,z_0,v_x,v_y,v_z)$, which we will use symmetries to reduce down to three.

First we use shift symmetries in $x$ and $t$ to set $x_0=0$. This simply means that we define $t=0$ to be the time when the light ray is closest to the lens.
Next, we can set $v_x=0$. 
This means that instead of letting the lens to have an $x$-velocity, the asymptotic observer who measures the polarization will have a nonzero $x$-velocity. 
This changes nothing because the light ray is in the $x$ direction, $k^\mu = (1,1,0,0)$.
Independent of what $x$-velocity the observer has, the plane orthogonal to both the light ray and the observer will be the $y$-$z$ plane.
Thus we are always calculating the rotation of polarization on the $y$-$z$ plane.
Finally, using rotational symmetry on the $y$-$z$ plan, we can set $v_z=0$, thus the remaining three parameters are $v_y=v$, $y_0$ and $z_0$. The required coordinate transformation from Eq.~(\ref{eq-SchIso}) is given by
\begin{eqnarray}
\gamma = (1-v^2)^{-1/2}~, \ \ t \rightarrow \gamma(t-vy)~, \ \ y\rightarrow \gamma(y-vt-y_0)~, \ \
z\rightarrow (z-z_0)~.
\end{eqnarray}
The resulting metric becomes
\begin{eqnarray}
g_{ab}dx^adx^b &=& -\left[1-\gamma^2(1+v^2)\frac{2M}{r}\right]dt^2   
+ \left[1+\gamma^2(1+v^2)\frac{2M}{r}\right]dy^2 
\label{eq-metric}
\\ \nonumber
& & - \frac{8Mv\gamma^2}{r}dtdy + \left(1+\frac{2M}{r}\right)(dx^2+dz^2)~, \\
r &=& \sqrt{ \gamma^2(y-vt-y_0)^2 + x^2 + (z-z_0)^2 }~.
\end{eqnarray}

While calculating the connections,
\begin{equation}
\Gamma_{ab}^c = \frac{g^{cd}}{2}\left(\partial_a g_{bd} + \partial_b g_{ad} - \partial_d g_{ab} \right)~,
\end{equation}
we can treat the first $g^{cd}$ as the flat metric $\eta^{cd}$ since we are only keeping the leading order result. 
This applies to any $g^{ab}$ that is not hit by a derivative in the calculation, for example the one in Eq.~(\ref{eq-Delta}).
We also assume that both the null ray direction and the polarization direction are only changed by a small amount (the Born approximation). 
Thus we can compute $\Delta_{ab}$ by Eq.~(\ref{eq-Delta}) along the undeflected light ray $x=t$.
Many components of $g_{ab}$ are zero due to our symmetry choice, so it is straightforward to see that
\begin{equation}
\Delta_{zy} = -\Delta_{yz} = \frac{1}{2}\int_{-\infty}^{\infty} dt \partial_z g_{ty}
=-2Mv\gamma^2 z_0 \int_{-\infty}^{\infty}
\frac{dt}{\left[\gamma^2(vt+y_0)^2+t^2 + z_0^2\right]^{3/2}}~.
\end{equation}

For $v\ll1$, we can perform the integral and keep only the leading order value to get 
\begin{equation}
\Delta\phi \equiv |\Delta_{yz}| \approx \frac{4Mvz_0}{y_0^2 + z_0^2}
= \frac{4|J_x|}{r^2}~.
\end{equation}
Here $J$ is the lens' angular momentum component in the $x$ direction---the direction of the light ray, with respect to $x=0$---the point where the light ray was closest to the lens. 
Such generalization follows directly from our symmetry choice.

Furthermore, assume that many light rays keep coming in the $x$ direction while the lens is moving in the constant velocity.
Then we will get maximal rotation at the light ray which is closest to the lens, which corresponds to $y_0=0$ in the above calculation.
\begin{equation}
\Delta \phi_{Max} = \frac{4M}{z_0}v = \frac{4M}{r}v~.
\end{equation}
This is the promised result in Eq.~(\ref{eq-v}).

\subsection{General Case}

The above point-mass calculation assumes that it carries no spin. 
Many papers employed a Kerr metric instead to calculate how the angular momentum from the spin also contributes to the rotation of polarization. 
In the limit of small rotations, we can instead generalize the above result without explicitly using the Kerr metric. 
That is because Eq.~(\ref{eq-metric}) allows superposition when all lenses are not moving too fast and not too close to the light ray. 
The metric of multiple moving point masses are given by
\begin{eqnarray}
g_{ab}dx^adx^b &=& -\left( 1 - 2\sum_n \frac{m^{(n)}}{r^{(n)}} \right)dt^2
+\left( 1 + 2\sum_n \frac{m^{(n)}}{r^{(n)}} \right)(dx^2+dy^2+dz^2)
\\ \nonumber 
& & -8\sum_i \frac{m^{(n)} }{r^{(n)}}
\left(v^{(n)}_x dx + v^{(n)}_y dy+ v^{(n)}_z dz\right)~, \\
r^{(n)} &=& \sqrt{\left(x - x_0^{(n)} - v^{(n)}_x t\right)^2 + \left(y - y_0^{(n)} - v^{(n)}_y t\right)^2 
+ \left(z - z_0^{(n)} - v^{(n)}_z t\right)^2}~.
\end{eqnarray}

Their contributions to the total rotation also superimpose linearly.
\begin{equation}
\Delta\phi = 4 \left| \sum_n \frac{v_y^{(n)}z_0^{(n)} - v_z^{(n)}y_0^{(n)}}
{\left(y_0^{(n)}\right)^2+\left(z_0^{(n)}\right)^2} \right|
\label{eq-combine}
\end{equation}

This provides the general answer to any mass and velocity distribution. 
By the uniqueness theorem, the effect from a Kerr metric of mass $M$ and spin $S$ can be mimicked by a two-particle system at the leading order.
\begin{eqnarray}
v_z^{(1)} &=& v_z^{(2)} = 0~,  \ \ \ v_y^{(1)} = v -\delta v~, \ \ \ v_y^{(2)} = v + \delta v~, \nonumber \\
y_0^{(1)} &=& y_0^{(2)} = y_0~, \ \ \ z_0^{(1)} = z_0-d~, \ \ \ z_0^{(2)} = z_0 + d~, \\
m^{(1)} &=& m^{(2)} = M/2~, \ \ \ S = M (\delta v) d~. \nonumber
\end{eqnarray}

\begin{eqnarray}
\Delta \phi = \frac{vz_0 - \delta v z_0 - vd}{y_0^2 + z_0^2 - 2z_0d}
\end{eqnarray}


\section{Example: Double Pulsar}
\label{sec-prediction}


\section{Known Disagreement/Confusions}



\bibliography{all_active}


\end{document}
